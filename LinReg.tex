\documentclass[a4paper, 12pt]{scrartcl}
\usepackage[brazil]{babel}
\usepackage{cmap}
\usepackage[T1]{fontenc}
\usepackage[utf8]{inputenc}
\usepackage{indentfirst}
\usepackage{amsmath,amsthm,amssymb}
\setlength{\parindent}{1cm} % padrão 15pt.
\usepackage{mathtools}
\usepackage{graphicx}
\usepackage{anysize}
\usepackage{latexsym,dsfont}
\usepackage{multicol}
\usepackage{enumerate}
\usepackage[usenames,dvipsnames,svgnames,table]{xcolor}
\usepackage{caption}
\usepackage[brazilian]{cleveref}
\theoremstyle{plain}% default
\newtheorem{teo}{Teorema}
\newtheorem{lema}[teo]{Lema}
\newtheorem{prop}[teo]{Proposição}
\newtheorem{cor}[teo]{Corolário}

\usepackage{setspace}
\theoremstyle{definition}
\newtheorem{defin}[teo]{Definição}
\newtheorem{conj}{Conjectura}
\newtheorem{exmp}{Exemplo}

\theoremstyle{remark}
\newtheorem*{obs}{Observação}
\newtheorem*{nota}{Nota}
\newtheorem{case}{Case}

% \usepackage[style=abnt]{biblatex}
% \addbibresource{bibliography.bib}

\def \Real {\mathds{R}}
\def \Natural {\mathds{N}}
\DeclareMathOperator{\mdc}{mdc}

\title{Regressão Linear}

\author{Martin Lobe \\ Orientador: Luiz Rafael dos Santos}

\date{}
\begin{document}
\onehalfspacing
\maketitle
 
\section*{Introdução}

Em regressão, buscamos uma função $f$ da qual mapeia pontos em $\Real^D$ para valores correspondentes em $f(x) \in R$
O objetivo é criar um modelo que a partir da informação de treinamento, consiga generalizar o suficiente para prever os resultados de novos pontos de entrada.



\section{Formulando o problema}
Abordaremos o problema de forma probabilistica. O motivo para isso é devido ao \textit{Observation Noise}.
\textit{Observation Noise} é a variabilidade do erro que possui em relação aos valores observaos e as funções que estão por trás desses medidas feitas. 
Estamos considerando que essas variabiliddades são igualmente e independentemente distribuidas
\[
    p(y|x) = \mathcal{N}(y|f(x), \sigma^2)
\]
(Estamos comparando a aproximação de $p$ com a função gaussiana sendo $\mathcal{N}$)
\\
Na equação mostrada, temos que $x \in \Real^D$, os nossos valores recebidos, enquanto $y \in \Real$ é o valor resultado pela função nesse ponto, sendo o nosso objetivo.

\[
    y = f(x)+\epsilon ; \epsilon ~ \mathcal{N}(0,\epsilon^2)
\]
Como podemos ver abaixo, f é a função que estamos querendo descobrir, com o valor de y sendo próximo dele. Essa proximidade é uma probabilidade dada pela função Gaussiana.
Essa função f é desconhecida, e o objetivo é encontrar uma função que seja próximo o suficiente dela.

Vamos começar com funções paramétricas. Começaremos considerando que a variancia $\sigma^2$ é conhecida, e trabalharemos em encontrar um parametro $\theta$

\begin{exmp}[Exemplo]\label{exmp: B} 
Para $ x,\theta \in \Real $, o modelo de regressão linear se torna uma reta, ou seja uma função linear.
O parametro $\theta $ é a inclinação dessa reta. .
%talvez ver alguma ligação com derivada??
\end{exmp}




\begin{exmp}[Exemplo]\label{exmp: B} 
\end{exmp}


% \printbibliography
\end{document}